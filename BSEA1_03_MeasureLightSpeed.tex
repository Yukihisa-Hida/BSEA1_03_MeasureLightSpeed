\documentclass{jarticle}

\usepackage[dvipdfmx]{graphicx}
\usepackage{float}
\usepackage{url}

\title{光速度の測定}
\author{2511198 肥田幸久}
\date{2025年5月22日}

\begin{document}
\maketitle



\section{実験の目的}
光速度は物理学のもっとも基本的な定数のひとつであり, 国際単位系の$\mathrm{m}$の定義にも採用されている. 本実験ではこの光速度と, 同軸ケーブルを信号が伝わる速度の測定を行う.


\section{実験の原理}
光速度の測定には, 木星の衛星の食, 光行差, 回転歯車, 回転鏡など歴史的にさまざまな方法が用いられてきた. 本実験では, それらに比べてシンプルかつ直接的な方法である「距離と時間の差」により、光速度を測定する.

光速度$c$は, 距離$d$を光が移動するのにかかる時間$t$を用いて,
\begin{equation}
  c=\frac{d}{t}
\end{equation}
と表される. この関係は本実験のもっとも根本的な原理である.

本実験では, 同一のパルスレーザー光を短距離経路と長距離経路に分岐させ, それぞれの往復にかかる時間を比較する. 両者の時間差$T$と距離差$L$を用いて光速度$c$は以下のように表される.
\begin{equation}
  c=\frac{L}{T}
\end{equation}
これにより, 絶対的な時間や距離を測定することなく, 高精度な光速度の測定が可能となる.



\section{実験方法}



\section{実験結果}



\section{考察}



\begin{thebibliography}{99}


\end{thebibliography}


\end{document}